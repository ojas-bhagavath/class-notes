\lecture{02}{10 Jan 2024 11:30}{Introduction to Banach Spaces}
\begin{definition}[Normed Linear Space]    Consider a vector space $V$ over a field $\mathbb{F}$ (either $\mathbb{R}$ or $\mathbb{C}$). A function $\|\cdot\|:V\to[0,\infty)$ is said to be a norm if it satisfies the following axioms:
    \begin{enumerate}
        \item [N1.] $\|x\|=0$ if and only if $x=0$.
        \item [N2.] $\|\alpha x\|=|\alpha|\|x\|$ for all $\alpha\in\mathbb{F}$ and $x\in V$.
        \item [N3.] $\|x+y\|\leq \|x\|+\|y\|$ for all $x,y\in V$.
    \end{enumerate}
    Then the vector space equipped with such a norm is called as a normed linear space.
\end{definition}
\vspace{0.4cm}
\begin{motive}
    When a norm is defined on a vector space, it induces a topology (metric topology) on the vector space. With a topological structure on $V$, we can do further analysis in $V$ such as limits of sequences, convergence, continuity of functions, etcetera.
\end{motive}
\vspace{0.4cm}
\begin{remark}[]
    If $(V,\|\cdot\|)$ is a normed linear space, then $d(x,y)=\|x-y\|:V\times V\to[0,\infty)$ is a metric on $V$:
    \begin{enumerate}
        \item [M1.] $d(x,y)=0\iff \|x-y\|=0\iff x-y=\iff x=y$
        \item [M2.] $d(x,y)=|-1|\|x-y\| = \|y-x\|=d(y,x)$
        \item [M3.] $
            d(x,z)=\|x-z\| = \|x-y+y-z\|
            \leq \|x-y\|+\|y-z\|
            =d(x,y)+d(y,z)$
    \end{enumerate}
    Equipped with this metric, $(V,d)$ is a metric space, and we can define an open subset of $V$ as follows:\\ 
    $U\subseteq V$ is said to be open if for all $x$ in $U$, there exists an $r>0$ such that $$B(x,r):=\{y\in V\mid d(x,y)<r\}\subseteq U$$
    The collection of these open subsets defines a topology on $V$.
\end{remark}
\vspace{0.4cm}
\begin{recall}
    \begin{itemize}
        \item []
        \item In a normed linear space $(V,\|\cdot\|)$, if a sequence $x_{n}$ converges to $x$ and another sequence $y_{n}$ converges to $y$, then the sequence $x_{n}+y_{n}$ converges to $x+y$.
        \item If $\alpha_{n}$ is a sequnce in $\mathbb{F}$ that converges to $\alpha$ and $x_{n}$ is a sequence in $V$ that converges to $x$, then the sequence $\alpha_{n}x_{n}$ converges to $\alpha x$.
        \item Therefore, addition and scalar multiplication are continuous maps in the topological spaces induced by a norm.
    \end{itemize}
\end{recall}
\vspace{0.4cm}
\begin{motive}
    Now that we have a metric space structure on $V$, we can talk about completeness (every Cauchy sequence converges).
\end{motive}
\vspace{0.4cm}
\begin{definition}[Banach Spaces]
    A normed linear space $(V,\|\cdot\|)$ is said to be a Banach Space if and only if $(V,d)$ is a complete metric space. That is, every Cauchy sequence in $V$ converges in the metric space $(V,d)$.
\end{definition}
\vspace{0.4cm}
\begin{recall}
    \begin{itemize}
        \item []
        \item \textbf{H\"older's inequality:}\\ 
            \hypertarget{holder}{Let $p,p*\in[1,\infty]$ such that $\frac{1}{p}+\frac{1}{p*}=1$ (such a pair of real numbers is called \textbf{conjugate exponents} of each other). For a $p\in[1,\infty)$, define a function from $\mathbb{R}^{N}$ to $\mathbb{R}$ as: $$\|x\|_{p}=\left(\sum_{i=1}^{N}|x_{i}|^{p}\right)^{\frac{1}{p}}$$ and for $p=\infty$, define another a function from $\mathbb{R}^{N}$ to $\mathbb{R}$ as: $$\|x\|_{\infty}=\max_{1\leq i\leq N}(x_{i})$$
                Let $x=(x_{1},\cdots,x_{N}),y=(y_{1},\cdots,y_{N})\in \mathbb{R}^{N}$.\\
            Then the following inequality holds: $$\sum_{i=1}^{N}|x_{i}y_{i}|\leq\|x\|_{p}\|y\|_{p*}$$}
        \item \textbf{Minkowski's inequality:}\\ 
            \hypertarget{minkowski}{For any $p\in[1,\infty]$, and $x,y\in \mathbb{R}^{N}$, we have the following inequality: $$\|x+y\|_{p}\leq\|x\|_{p}+\|y\|_{p}$$}
        \item \textbf{Jensen's Inequality}:\\ 
            \hypertarget{jensen}{For any pair of real numbers $x,y\in \mathbb{R}$, the following inequality holds: $$|a+b|^{p}\leq2^{p-1}(a^{p}+b^{p})$$}
    \end{itemize}

\end{recall}

\begin{eg}[$p$-norm on $\mathbb{R}^{N}$]
    For any $p\in[1,\infty]$, $(\mathbb{R}^{N},\|\cdot\|_{p})$ is a Banach space.
    \begin{proof}
        First we need to prove that $\|\cdot\|_{p}$ is a norm on $\mathbb{R}^{N}$.\\ 
        This follows from the \hyperlink{minkowski}{Minkowski's inequality}.
        \begin{note}
            This is usually called as $p$-norm on $\mathbb{R}^{N}$, and the special case when $p=2$ is called the Euclidean norm.
        \end{note}
        Then we need prove the completeness of $(\mathbb{R}^{N},\|\cdot\|_{p})$. First we prove for $p\in[1,\infty)$ case.\\
        Suppose $x^{(n)}$ is a Cauchy sequence in the NLS. Then for each $\epsilon>0$, there exists a natural number $t$ such that $\|x^{(n)}-x^{(m)}\|_{p}<\epsilon$ whenever $m,n>t$. That is, $m,n>t$ implies that $$\sum_{i=1}^{N}\left|x^{(n)}_{i}-x_{i}^{(m)}\right|^{p}<\epsilon^{p}$$
        Hence $\left|x_{i}^{(n)}-x_{i}^{(m)}\right|<\epsilon$ whenever $m,n>t$. That means $x_{i}^{(n)}$ is a Cauchy sequence in $\mathbb{R}$, hence converges (since $\mathbb{R}$ is complete) to a point, say $x_{i}\in \mathbb{R}$.\\
        Let $x=(x_{1},x_{2},\cdots,x_{N})\in \mathbb{R}^{N}$, where each $x_{i}$ are defined as above. We claim that $x^{(n)}$ converges to $x$.\\ 
        We have the inequality $|x_{i}^{(n)}-x_{i}^{(m)}|<\epsilon^{p}$ whenever $m,n>t$.\\ 
        Fix $n$, and take the limit as $m\to\infty$, we obtain $|x_{i}^{(n)}-x_{i}|\leq\epsilon^{p}$.\\ 
        Therefore, after an appropriate adjustment, we obtain an $\epsilon'$, such that for any $\epsilon>0$, there exists a natural number $t$ such that $n>t$ implies that the finite sum satisfies the inequality $$\left(\sum_{i=1}^{N}\left|x_{i}^{(n)}-x_{i}\right|^{p}\right)^{\frac{1}{p}}=\|x^{(n)}-x\|_{p}<\epsilon'$$
        Hence $x^{(n)}$ converges to $x$.
        Therefore, every Cauchy sequence in $(\mathbb{R}^{N},\|\cdot\|_{p})$ converges, hence it is a Banach space when $p\in[1,\infty)$.\\ 
        When $p=\infty$, as convergence and Cauchy criterion hold if and only if they hold componentwise in $\mathbb{R}^{N}$, $(\mathbb{R}^{N},\|\cdot\|_{\infty})$ is Banach as well.
    \end{proof}
\end{eg}
\begin{note}
    Due to completeness of $\mathbb{C}$, for any $p\in[1,\infty)$, $(\mathbb{C},\|\cdot\|_{p})$ is a Banach space as well.
\end{note}
\vspace{0.4cm}
\begin{eg}[$\bm{\ell_{p}}$ \textbf{spaces}]
    For $1\leq p<\infty$, define a collection of sequences as follows:$$\ell_{p}:=\left\{x=(x_{1},x_{2},\cdots)\mid x_{i}\in\mathbb{F}, \sum_{i=1}^{\infty}|x_{i}|^{p}<\infty\right\}$$
    Now, define a function from $\ell_{p}$ to $\mathbb{F}$ as $\|x\|_{p}=\left(\sum_{i=1}^{\infty}|x_{i}|^{p}\right)^{\frac{1}{p}}$. Then $(\ell_{p},\|\cdot\|_{p})$ is a Banach space.
    \begin{proof} The proof proceeds in several steps:
        \begin{enumerate}
            \item $\bm{\ell_{p}}$ \textbf{is a vector space.} 
                \begin{proof}
                    Let $x,y\in\ell_{p}$, then from the Jensen's inequality, we have $$|x_{i}+y_{i}|^{p}\leq(|x_{i}|+|y_{i}|)^{p}\leq 2^{p-1}(|x_{i}|^{p}+|y_{i}|^{p})<\infty$$
                    and for any $\alpha\in \mathbb{F}$ and $x\in\ell_{p}$, $\alpha x$ clearly belongs to $\ell_{p}$.\\ 
                    Hence $\ell_{p}$ is a subspace of the space of all sequences in $\mathbb{F}$.
                \end{proof}
            \item $\bm{\|\cdot\|_{p}}$\textbf{ is a norm on }$\bm{\ell_{p}}$\textbf{.}
                \begin{proof}
                    The first two axioms of a norm are trivial to prove. We need to show the triangle inequality.\\ 
                    Let $x,y\in\ell_{p}$, then for any natural number $N$, from \hyperlink{minkowski}{Minkowski's Inequality}, we have $$\sum_{i=1}^{N}|x_{i}+y_{i}|^{p}\leq\left[\left(\sum_{i=1}^{N}|x_{i}|^{p}\right)+\left(\sum_{i=1}^{N}|y_{i}|^{p}\right)\right]^{p}\leq\left[\|x\|_{p}+\|y\|_{p}\right]^{p}<\infty$$
                    Since $N$ is arbitrary, and the right side is independent of the choice of $N$, it follows that $$\sum_{i=1}^{\infty}|x_{i}+y_{i}|^{p}\leq\left[\|x\|_{p}+\|y\|_{p}\right]^{p}<\infty$$
                    Hence the triangle inequality holds, and hence $\|\cdot\|_{p}$ is a norm on $\ell_{p}$.
                \end{proof}
            \item $\bm{(\ell_{p},\|\cdot\|_{p})}$\textbf{ is a complete space.}
                \begin{proof}
                    Let $x^{(n)}$ be a Cauchy sequence in the NLS, that is, for each $\epsilon>0$, there exists a natural number $t$ such that $\|x_{i}^{(n)}-x^{(m)}\|_{p}<\epsilon$ whever $m,n>t$. That is, $m,n>t$ implies that $$\sum_{i=1}^{\infty}\|x_{i}^{(n)}-x_{i}^{(m)}\|<\epsilon^{p}$$ 
                Hence $\left|x_{i}^{(n)}-x_{i}^{(m)}\right|<\epsilon$ whenever $m,n>t$, hence $x_{i}^{(n)}$ is a Cauchy sequence in $\mathbb{R}$, hence converges (since $\mathbb{R}$ is complete) to a point, say $x_{i}\in \mathbb{R}$.\\ 
                Construct a sequence $x=(x_{1},x_{2}\cdots x_{n},\cdots)$ where $x_{i}$ is defined as above.\\
                We need to show that $x\in\ell_{p}$. Since $x^{(n)}$ is a Cauchy sequence, it is bounded, hence there exists a constant $C>0$ such that $$\|x^{(n)}\|_{p}^{p}=\sum_{i=1}^{\infty}\left|x_{i}^{(n)}\right|^{p}\leq C\leq C,~\forall n.$$
                Let $k$ be any fixed positive integer. Then, $$\sum_{i=1}^{k}\left|x_{i}^{(n)}\right|^{p}\leq C$$ which implies that $$\sum_{i=1}^{k}\left|x_{i}\right|^{p}\leq C.$$
                Since $k$ is arbitrary and the right side does not depend on it, it follows that $$\sum_{i=1}^{\infty}\left|x_{i}\right|^{p}\leq C<\infty.$$
                Hence $x\in\ell_{p}$.\\ 
                Now we need to show that $x^{(n)}$ converges to $x$. From before, whenever $m,n>t$, we have $$\sum_{i=1}^{\infty}|x_{i}^{(n)}-x_{i}^{(m)}|^{p}<\epsilon^{p}.$$
                Hence, for any positive integer $k$, and whenever $m,n>t$, we have $$\sum_{i=1}^{k}|x_{i}^{(n)}-x_{i}^{(m)}|^{p}<\epsilon^{p}.$$
                Fix $n>t$, and take the limit $m\to\infty$, we obtain $$\sum_{i=1}^{k}|x_{i}^{(n)}-x_{i}|^{p}<\epsilon^{p}.$$
                Since $k$ is arbitrary and the right side does not depend on it, hence whenever $n>t$, we have $$\|x^{(n)}-x\|_{p}\leq \epsilon.$$
                That is, $x^{(n)}$ converges to $x$.
                Hence every Cauchy sequence in $(\ell_{p},\|\|_{p})$ converges, hence it is Banach.
                \end{proof}
        \end{enumerate}
        \phantom\qedhere
    \end{proof}
\end{eg}
\vspace{0.4cm}
\begin{eg}
    Consider the set of sequences $$\ell_{\infty}:=\left\{x=(x_{i})\mid x_{i}\in \mathbb{F}, \sup_{1\leq i<\infty}|x_{i}|<\infty\right\},$$ clearly it is a vector space under componentwise addition and scalar multiplication. Also define the function $$\|x\|_{\infty}=\sup_{1\leq i<\infty}|x_{i}|$$. Then $\|x\|_{\infty}$ is a norm on $\ell_{\infty}$ and $(\ell_{\infty},\|\cdot\|_{\infty})$ is a Banach space.
\end{eg}
\vspace{0.4cm}

