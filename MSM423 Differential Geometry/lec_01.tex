\lecture{01}{05 Jan 2024 10:30}{Introduction to Curves and Surfaces}

\begin{definition}[Curves]
    Let $I$ be an open interval in $\mathbb{R}$. A smooth parametrized curve in $\mathbb{R}^n$ is a smooth map $\gamma:I\to\mathbb{R}^n$.\\
    The image $\gamma(I)\subseteq\mathbb{R}^n$ is called trace of the curve $\gamma$.
\end{definition}
\vspace{0.4cm}
\begin{eg}
    $\gamma_1,\gamma_2:\mathbb{R}\to\mathbb{R}^2$, defined as $\gamma_1(t)=(\cos(t),\sin(t))$ and $\gamma_2(t)=(\sin(3t),\cos(3t))$ have the same trace, but they are different curves.
\end{eg}
\vspace{0.4cm}
\begin{observe}
    Consider any circle in $\mathbb{R}^2$, it is the set of all points in $\mathbb{R}^2$ satisfying a certain quadratic equation.\\
    It can also be viewed as a trace of some curve.
\end{observe}
\vspace{0.4cm}
\begin{definition}[Level Set]
    Let $U\subseteq\mathbb{R}^n$ be any open set, and $f:U\to\mathbb{R}$ be any function.\\
    Then for a given constant $c\in\mathbb{R}$, the level set is defined as$$L_{c}(f):=\{X \in U \mid f(X)=c\} \subseteq U \subseteq\mathbb{R}^n$$
\end{definition}
\vspace{0.4cm}
\begin{eg}
    Let $U=\mathbb{R}^2$, and $f(x,y)=x^2+y^2$, then\\ 
    $L_{0}(f)={(0,0)}$, a point,\\
    $L_{1}(f)=S^{1}=\{(x,y)\in\mathbb{R}^2\mid x^2+y^2=1\}$, a circle in $\mathbb{R}^2$,\\
    $L_{-1}(f)=\varnothing$.
\end{eg}
\vspace{0.4cm}
\begin{definition}[Graph]
    Let $A$ and $B$ be any sets. Let $f:A\to B$ be a function, then the graph of $f$ is the function $G_{f}:A\to A\times B$, given by$$G_f(x)=(x,f(a))\in A\times B$$
\end{definition}
\vspace{0.4cm}
\begin{note}
    Notice that a graph is also a curve, and if $f$ is a smooth function, then $G_f$ is a smooth curve.
\end{note}
\vspace{0.4cm}
\begin{theorem}[Implicit Function Theorem (2D Case)]
    Let $F:\mathbb{R}^2\to\mathbb{R}$ be a continuous function defining a curve $F(x,y)=c$.\\ 
    Let $(x_0,y_0)$ be a point on the curve.
    \begin{itemize}
        \item If $\frac{\partial F}{\partial x}(x_0,y_0)\neq 0$, then there exists a neighborhood around $(x_0,y_0)$ where we can write $y=f(x)$ for a real valued function $f(x)$. That is, the curve $F(x,y)=c$ behaves like the graph of $y=f(x)$ in that neighborhood.
        \item If $\frac{\partial F}{\partial y}(x_0,y_0)\neq 0$, then there exists a neighborhood around $(x_0,y_0)$ where we can write $x=f(y)$ for a real valued function $f(x)$. That is, the curve $F(x,y)=c$ behaves like the graph of $x=g(x)$ in that neighborhood.
    \end{itemize}
\end{theorem}
\vspace{0.4cm}
\begin{eg}
    Consider $S^1=L_1(x^2+y^2)=\{(x,y)\mid x^2+y^2=1\}$\\ 
    Here, $F(x,y)=x^2+y^2$, and $\frac{\partial F}{\partial x}=2x\neq 0$ when $x\neq0$.\\ 
    Therefore, for any point $(x_0,y_0)$ in $S^1$, where such that $x_0\neq0$, there exists a neighborhood in which $x=g(y)$.
\end{eg}
\vspace{0.4cm}
\begin{note}
    It is worthwhile to notice that a graph is always a curve. Implicit function theorem gives a condition on when a curve (level set in particular) can be seen as the graph of a curve.
\end{note}
\vspace{0.4cm}

