% Edit the maketitle command
\usepackage{titling} % Enables editing the \maketitle command
\renewcommand{\maketitle}{
    \begin{center}
        \Huge\bfseries
        \thetitle\\
        \vspace{4mm}
        \Large
        \theauthor
    \end{center}
    \vspace{1.5cm}
}
\author{Ojas G Bhagavath}

\usepackage{xifthen} % Extended conditional commands
\makeatother
\def\@lecture{}%
\newcommand{\lecture}[3]{
    \ifthenelse{\isempty{#3}}{%
        \def\@lecture{Lecture #1 \hfill #2\vspace{0.4cm}}%
    }{%
        \def\@lecture{Lecture #1: #3 \hfill #2\vspace{0.4cm}}%
    }%
    \subsection*{\@lecture}
}
\makeatletter

\usepackage[backend=biber]{biblatex}
\usepackage[colorlinks=true,linkcolor=blue,citecolor=red,urlcolor=blue]{hyperref}
\usepackage[utf8]{inputenc} % Accept different input encodings
\usepackage[T1]{fontenc} % Standard package for selecting font encodings
\usepackage{textcomp} % LATEX support for the Text Companion fonts
\usepackage{url} % Verbatim with URL-sensitive line breaks
\usepackage{graphicx} % Enhanced support for graphics
\usepackage{float} % Improved interface for floating objects
\usepackage{booktabs} % Publication quality tables in LaTeX
\usepackage{enumitem} % Control layout of itemize, enumerate, description
\usepackage{parskip} %Layout with zero \parindent, non-zero \parskip
\usepackage{emptypage} % Make empty pages really empty
\usepackage{subcaption} % Support for sub-caption
\usepackage{multicol} % Intermix single and multiple columns
\usepackage{xcolor} % Driver-independent color extensions for LATEX and pdfLATEX
\usepackage{amsmath, amsfonts, mathtools, amsthm, amssymb, mathrsfs} % Packages related to math
\usepackage{cancel} % Place lines through maths formulae
\usepackage{bm} % Access bold symbols in math mode
\usepackage{systeme} % Format systems of equations
\usepackage{mdframed} %Framed environments that can split at page boundaries
\usepackage{fancyhdr} % For fancy headers and footers
\usepackage{todonotes} % todonotes and inline notes in fancy boxes
\usepackage{tcolorbox} % Colored boxes for LaTeX
\usepackage{import} % Establish input relative to a directory
\usepackage{pdfpages} % Include PDF documents in LaTeX
\usepackage{transparent} % Using a color stack for transparency with pdfTEX
\usepackage{array}
\usepackage{tabularx}
\usepackage{varwidth}
\usepackage[margin=0.75in]{geometry}

% Some shortcuts
\newcommand\N{\ensuremath{\mathbb{N}}}
\newcommand\R{\ensuremath{\mathbb{R}}}
\newcommand\Z{\ensuremath{\mathbb{Z}}}
\renewcommand\O{\ensuremath{\varnothing}}
\newcommand\Q{\ensuremath{\mathbb{Q}}}
\newcommand\C{\ensuremath{\mathbb{C}}}


\newtheoremstyle{unbreak}%
{}
{}%
{\upshape}
{}%
{\bfseries}
{:}% % Note that final punctuation is omitted.
{0.5em}
{}

\newtheoremstyle{break}%
{}%
{}%
{\upshape}
{}%
{\bfseries\itshape}
{:}% % Note that final punctuation is omitted.
{\newline}
{}

\theoremstyle{break}
\newmdtheoremenv[nobreak=true]{theorem}{Theorem}
\newmdtheoremenv[nobreak=true]{definition}{Definition}
\newmdtheoremenv[nobreak=true]{prop}{Proposition}
\newmdtheoremenv[nobreak=true]{corollary}{Corollary}
\newtheorem*{recall}{Recall}

\theoremstyle{unbreak}
\newtheorem*{eg}{Example}
\newtheorem*{motive}{Motivation}
\newtheorem*{remark}{Remark}
\newtheorem*{note}{Note}
\newtheorem*{problem}{Problem}
\newtheorem*{observe}{Observe}
\newtheorem*{property}{Property}
